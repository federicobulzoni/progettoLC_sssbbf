\subsection{Il modulo di analisi statica}
In questa sezione viene fornita una breve descrizione della logica retrostante il funzionamento del modulo di analisi statica \texttt{StaticAnalysis.hs}.

Il modulo di analisi statica fornisce un'interfaccia alle restanti componenti del compilatore, che permette di ottenere un albero di sintassi astratta annotato ed eventualmente una lista di log a partire da un programma \SBF\ rappresentato in sintassi astratta.

Per la gestione dei log si rimanda alla sezione riguardante le tecniche non standard utilizzate.

Il modulo è stato progettato seguendo un approccio \textit{top-down}, andando di volta in volta a risolvere i sotto-problemi in modo ricorsivo.

Il modulo di analisi statica fa largo uso del modulo \textit{Environment.hs} che gli offre un'interfaccia per la gestione dell'environment del programma sotto analisi. 

L'environment di un programma è una pila di \textit{scope}, dove ogni scope è formato da: una \textit{tabella di lookup} che mette in corrispondenza gli identificatori definiti all'interno dello scope con le informazioni a riguardo, un \textit{tipo} che eredita dalla funzione in cui lo scope è stato creato ed infine un \textit{valore booleano} che indica se nello scope è presente o meno  un \texttt{return} con tipo compatibile con quello dello scope.

Il modulo \texttt{Environment.hs} fornisce tutte le funzioni necessarie per la gestione dell'environment di un programma. La funzione \texttt{lookup}  permette di ottenere le informazioni di un identificatore precedentemente dichiarato nello scope corrente, o in un \textit{super-scope} nel quale lo scope corrente è contenuto; nel caso in cui l'identificatore cercato non sia trovato un eccezione viene ritornata.
La funzione \texttt{update} permette di inserire una nuova corrispondenza \textit{identificatore-info} nello scope corrente. Viene ritornata un'eccezione, nel caso in cui l'identificatore che si sta cercando di inserire fosse già stato dichiarato in precedenza all'interno dello scope.
Per ottenere il tipo dello scope corrente è presente la funzione \textit{getScopeType}, la funzione \textit{hasReturn} ritorna invece il valore booleano che indica se nello scope corrente è presente o meno un \texttt{return} di tipo coerente con quello dello scope; quando un tale \texttt{return} viene trovato la funzione \texttt{setReturnFound} permette di impostare tale valore booleano a \texttt{True}.
La funzione \texttt{addScope} aggiunge un nuovo scope all'environment, prendendo come argomento il tipo dello scope che si sta aggiungendo. Qui è importante porsi una domanda:
\begin{center}
Quale tipo viene passato ad \texttt{addScope}?
\end{center}
La risposta è: \textit{dipende}. 
Per convenzione, se lo scope è lo scope globale del programma, esso ha tipo \texttt{TSimple\_TypeVoid}; se lo scope viene creato da una dichiarazione di funzione, allora il tipo passato ad \texttt{addScope} è quello della funzione dichiarata. Infine, se lo scope viene aggiunto durante la creazione di un blocco di statements, allora il tipo passato ad \texttt{addScope} è quello dello scope in cui il blocco di statements è racchiuso.
Non è difficile vedere che tali accorgimenti verificano la proprietà stabilita per il tipo di uno scope, ossia che esso coincida con il tipo della funzione in cui è racchiuso.

Forniamo ora una breve panoramica del funzionamento del modulo di analisi statica \texttt{StaticAnalysis.hs}. La funzione principale del modulo è \texttt{typeCheck} che preso in input un programma rappresentato in sintassi astratta di \SBF, restituisce tale programma annotato ed una lista di log (eventualmente vuota).
Il compito di annotare gli elementi della sintassi astratta e di aggiungere eventuali log alla lista di log ogni qual volta venga rilevata una eccezione, è affidato alle funzioni \texttt{infer*} (per esempio: \texttt{inferDecl}, \texttt{inferStm}, \texttt{inferExp}, ecc.), tali funzioni preso un elemento della sintassi astratta effettuano i controlli per verificare che sia coerente con le specifiche del linguaggio; nel caso non lo siano, tramite la funzione \texttt{saveLog}, viene aggiornata la lista dei log. Al termine dei controlli l'elemento analizzato viene ritornato arricchito con annotazioni quali: le informazioni sugli identificatori utilizzati e il tipo delle espressioni coinvolte. 

Le funzioni \texttt{infer*} hanno tutte un comportamento naturalmente ricorsivo, si prenda per esempio il caso di una dichiarazione di variabile con assegnamento di un valore, per poter inferire la dichiarazione e determinare se è valida o meno è necessario inferire l'espressione che si sta cercando di assegnare alla variabile e verificare che i tipi siano compatibili.

È stato deciso che, nel caso in cui si stia inferendo un elemento che coinvolge un'espressione che è già stato verificato essere problematica, non vengono aggiunte al log nuove eccezioni riguardanti l'elemento analizzato. Questa scelta è giustificata dal desiderio di eliminare ridondanza negli errori, dato che un'espressione errata, potrebbe portare ad una lunga serie di errori a catena in tutti gli elementi in cui essa è contenuta.

Per ottenere questo comportamento, è stato inserito nella sintassi astratta un tipo interno \texttt{SType\_Error} che viene assegnato alle espressioni che generano un'eccezione, o alle espressioni le cui sotto-espressioni generano eccezioni. Se un qualsiasi elemento ha come sotto-espressione una espressione di tipo \texttt{SType\_Error} non vengono generate ulteriori eccezioni riguardo all'interazione di tale sotto-espressione con l'elemento considerato.

Le annotazioni aggiunte dal modulo di analisi statica riguardano esclusivamente le espressioni e i parametri attuali nelle chiamate di funzione/procedura. Le $R$-espressioni annotate sono identificate dagli elementi di sintassi astratta interni \texttt{ExpTyped} e le $L$-espressioni annotate sono identificate dagli elementi di sintassi astratta interni \texttt{LExpTyped}, entrambi gli elementi condividono la stessa struttura, un'espressione tipata contiene l'espressione, il suo tipo e la locazione in cui viene utilizzata. Una caratteristica ulteriore delle espressioni tipate, è che la locazione contenuta all'interno di un identificatore (presente in un'espressione) non è la locazione di utilizzo, che invece è in un campo apposito, ma è la locazione di dichiarazione dell'identificatore.
La locazione di dichiarazione di un identificatore utilizzato all'interno di una espressione risulta essere fondamentale per la successiva fase di generazione del codice TAC.

I parametri attuali annotati sono identificati dagli elementi di sintassi astratta interni \texttt{ParExpTyped} che oltre ad avere l'informazione della espressione passata come parametro contengono anche il tipo del corrispondente parametro formale nella funzione/procedura chiamata.

Il modulo \texttt{Typed.hs} contiene la definizione degli elementi tipati con le loro proprietà.



