\section{Struttura lessicale di \SBF}

\subsection*{Parole riservate}
Le parole riservate in \SBF sono le seguenti alfanumeriche
\begin{equation}
\False,\ \True,\ \If,\ \Else,\ \Do,\ \While,\ \Def,\ \Return,\ \Val,\ \Var
\end{equation}
\hl{andrebbero ordinate}
e i seguenti simboli
\begin{equation}
\texttt{\_},\ \texttt{:},\ \texttt{=},\ \texttt{=>},\ \texttt{<-},\ \texttt{<:},\ \texttt{<\%},\ \texttt{>:},\ \texttt{\#},\ \texttt{@}
\end{equation}
\hl{non serviranno tutti}

\subsection*{Identificatori}
%	Un identificatore $\token{Ident}$ è uno dei seguenti:
%	\begin{enumerate}
%		\item una lettera o il carattere '\_' seguiti da una sequenza arbitraria di lettere, cifre e del carattere '\_';
%		\item una sequenza di uno o più operatori (ad esempio '+', ':', '?'). Il simbolo '\$' è escluso;
%		\item identificatori del primo tipo, seguiti da '\_' e da un un identificatore del secondo tipo;
%		\item una stringa arbitraria racchiusa fra back-quotes (ad esempio `def`).
%	\end{enumerate}
%	Sono escluse le parole riservate.
%	\hl{Ci sarebbe la distinzione fra identificatori di costante e non.}

Un identificatore $\token{Ident}$ è una lettera o il carattere '\_' seguiti da una sequenza arbitraria di lettere, cifre e del carattere '\_'.

\subsection*{Caratteri di ritorno a capo}
Nel linguaggio \SBF\ le istruzioni possono terminare con ';' oppure '\textbackslash n'. 
\begin{equation*}
\token{Sep} ::= \texttt{;}\ |\ \token{NewLine}
\end{equation*}

Il carattere di ritorno a capo '\textbackslash n' è sintatticamente significativo, e corrisponde a un token $\token{NewLine}$ quando le seguenti condizioni sono soddisfatte:
\begin{itemize}
	\item il token immediatamente precedente può terminare una istruzione;
	\item il token immediatamente successivo può iniziare una istruzione;
	\item il carattere appare in una regione dove il new line è abilitato.
\end{itemize}
\hl{Complicate da spiegare, sono scritte qua. }
\url{https://scala-lang.org/files/archive/spec/2.13/01-lexical-syntax.html#identifiers}

\subsection*{Letterali}
Ci sono letterali per numeri interi, numeri in virgola mobile, singoli caratteri, booleani, stringhe. Essi seguono le convenzioni della maggior parte dei linguaggi di programmazione.
\begin{align*}
\token{Literal} ::=
& \token{Int}\\
& \token{Float}\\
& \token{Char}\\
& \token{Boolean}\\
& \token{String}
\end{align*}

\subsection*{Caratteri di spaziatura e commenti}
I token possono essere separati da caratteri di spaziatura o commenti.

I commenti sono di due tipi:
\begin{itemize}
	\item i commenti di una riga sono sequenze di caratteri che iniziano con \verb$\\$ e finiscono al termine della riga;
	\item i commenti multi-riga sono sequenze di caratteri che iniziano con \verb$/*$ e terminano con \verb$*/$. Possono essere annidati, ma in modo bilanciato. Ad esempio \verb$/*commento*/*/$ non è un commento valido.
\end{itemize}

\hl{Permettiamo le trailing commas? In caso va detto qua in una sottosezione che non diventano token.}

\section{Struttura sintattica di \SBF}
\begin{itemize}
\item Un \emph{programma} è una sequenza di dichiarazioni separate da token di separazione $\token{Sep}$.


\item Una \emph{dichiarazione} ha una delle seguenti forme:
\begin{itemize}
	\item \emph{Dichiarazione di variabili e valori.}
	\begin{align*}
	\token{Decl} ::=\ &\texttt{val}\ \token{Ident} \texttt{:}\ \token{TypeSpec}\ \texttt{=}\ \token{Expr}\\
	|\ &\texttt{val}\ \token{Ident} \texttt{:}\ \token{TypeSpec}\\
	|\ &\texttt{var}\ \token{Ident} \texttt{:}\ \token{TypeSpec}\ \texttt{=}\ \token{Expr}\\
	|\ &\texttt{var}\ \token{Ident} \texttt{:}\ \token{TypeSpec}
	\end{align*}
	dove
	\begin{align*}
	\token{TypeSpec} ::=\ &\token{SimpleType}\\
	|\ &\texttt{\&}\ \token{TypeSpec}\\
	|\ &\texttt{Array}\ \texttt{[}\ \token{TypeSpec}\ \texttt{]}\\
	\token{SimpleType} ::=\ &\texttt{bool}| \texttt{char}| \texttt{integer}| \texttt{float}| \texttt{string}
	\end{align*}
	\item \emph{Dichiarazione di funzioni/procedure.}
	\begin{align*}
	\token{Decl} ::=\ & \texttt{def}\ \token{Ident}\ \token{ParamClauses} \texttt{:}\ \token{TypeSpec}\ \texttt{=}\ \token{Expr}\\
	& \texttt{def}\ \token{Ident}\ \token{ParamClauses} \texttt{:}\ \token{TypeSpec}\ \texttt{=}\ \token{Block}
	\end{align*}
	L'elemento $\token{ParamClauses}$ è una sequenza, non vuota, di $\token{ParamClause}$:
	\begin{equation*}
	\token{ParamClause} ::=\ \texttt{(}\ \token{ListOfParameters}\ \texttt{)}
	\end{equation*}
	mentre $\token{ListOfParameters}$ è una sequenza, che può essere vuota, di elementi separati da virgola della forma
	\begin{equation*}
	\token{Parameter} ::=\ \token{Ident}\ \texttt{:}\ \token{TypeSpec}
	\end{equation*}
	
\end{itemize}

\item Un \emph{blocco} è una sequenza di istruzioni racchiuse fra parentesi graffe e separate da $\token{Sep}$.

\item Una \emph{istruzione} ha la forma:
\begin{align*}
\token{Stmt} ::=\ &\texttt{if}\ \texttt{(}\ \token{Expr}\ \texttt{)}\ \token{Stmt}\\
|\ &\texttt{if}\ \texttt{(}\ \token{Expr}\ \texttt{)}\ \token{Stmt}\ \texttt{else}\ \token{Stmt}\\
|\ &\texttt{if}\ \texttt{(}\ \token{Expr}\ \texttt{)}\ \token{Stmt}\ \token{Sep}\ \texttt{else}\ \token{Stmt}\\
|\ &\texttt{while}\ \texttt{(}\ \token{Expr}\ \texttt{)}\ \token{Stmt}\\
|\ &\texttt{do}\ \token{Stmt}\ \texttt{while}\ \texttt{(}\ \token{Expr}\ \texttt{)}\\
|\ &\texttt{do}\ \token{Stmt}\ \token{Sep}\ \texttt{while}\ \texttt{(}\ \token{Expr}\ \texttt{)}\\
|\ &\texttt{return}\\
|\ &\texttt{return}\ \token{Expr}\\
|\ &\token{Block}\\
|\ &\token{LExpr}\\
|\ &\token{LExpr}\ \token{AssignmentOp}\ \token{Expr}\\
|\ &\token{Ident}\ \token{ParamClauses}
\end{align*}

\item Le \emph{left expressions} permesse nel linguaggio hanno la seguente forma.
\begin{align*}
\token{LExpr} ::=\ &\token{Ident}\\
|\ &\token{LExpr}\ \texttt{(}\ \token{Expr}\ \texttt{)}\\
|\ &\texttt{*}\ \token{LExpr}\\
|\ &\texttt{\&}\ \token{LExpr}\\
|\ &\texttt{(}\ \token{LExpr}\ \texttt{)}
\end{align*}
\hl{Precedenza di puntatori e array access}

\item Le \emph{right expressions} permesse nel linguaggio hanno la seguente forma.
\begin{align*}
\token{Expr} ::=\ &\token{LExpr}\\
|\ &\token{Literal}\\
|\ &\token{Expr}\ \token{BinOp}\ \token{Expr}\\
|\ &\token{UnOp}\ \token{Expr}\\
|\ &\token{Ident}\ \token{Lists}\\
|\ &\texttt{(}\ \token{Expr}\ \texttt{)}
\end{align*}
\begin{align*}
\token{BinOp} ::=\ &\texttt{||} | \texttt{\&\&}|\texttt{+}|\texttt{-}|\texttt{*}|\texttt{/}|\texttt{\%}|\texttt{\^{}}|\texttt{==}|\texttt{!=}|\texttt{<}|\texttt{<=}|\texttt{>}|\texttt{>=}\\
\token{UnOp} ::=\ &\texttt{!}\ \texttt{-}
\end{align*}


\end{itemize}
