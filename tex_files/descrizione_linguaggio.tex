\documentclass[11pt,a4paper,italian]{article}

\usepackage[utf8]{inputenc}
\usepackage{babel}
\usepackage{color, soul}
\usepackage{hyperref}
\usepackage{amsmath}
\usepackage{amsfonts}
\usepackage{amssymb}
\usepackage{amsthm}

\title{Descrizione del linguaggio SBF}

% Nome del linguaggio.
\newcommand{\SBF}{\textbf{SBF}}

% Token
\newcommand{\token}[1]{\langle \mathit{#1} \rangle }

% Parole riservate
\newcommand{\False}{\texttt{False}}
\newcommand{\True}{\texttt{True}}
\newcommand{\If}{\texttt{if}}
\newcommand{\Else}{\texttt{else}}
\newcommand{\Do}{\texttt{do}}
\newcommand{\While}{\texttt{while}}
\newcommand{\Def}{\texttt{def}}
\newcommand{\Return}{\texttt{return}}
\newcommand{\Val}{\texttt{val}}
\newcommand{\Var}{\texttt{var}}

\begin{document}
	\maketitle
	
	\section{Struttura lessicale di \SBF}
	\subsection*{Parole riservate}
	Le parole riservate in \SBF sono le seguenti alfanumeriche
	\begin{equation}
	\False,\ \True,\ \If,\ \Else,\ \Do,\ \While,\ \Def,\ \Return,\ \Val,\ \Var
	\end{equation}
	\hl{andrebbero ordinate}
	e i seguenti simboli
	\begin{equation}
	\texttt{\_},\ \texttt{:},\ \texttt{=},\ \texttt{=>},\ \texttt{<-},\ \texttt{<:},\ \texttt{<\%},\ \texttt{>:},\ \texttt{\#},\ \texttt{@}
	\end{equation}
	\hl{non serviranno tutti}
	
	\subsection*{Identificatori}
	Un identificatore $\token{Ident}$ è uno dei seguenti:
	\begin{enumerate}
		\item una lettera o il carattere '\_' seguiti da una sequenza arbitraria di lettere, cifre e del carattere '\_';
		\item una sequenza di uno o più operatori (ad esempio '+', ':', '?'). Il simbolo '\$' è escluso;
		\item identificatori del primo tipo, seguiti da '\_' e da un un identificatore del secondo tipo;
		\item una stringa arbitraria racchiusa fra back-quotes (ad esempio `def`).
	\end{enumerate}
	Sono escluse le parole riservate.
	
	\hl{Ci sarebbe la distinzione fra identificatori di costante e non.}
	
	\subsection*{Caratteri di ritorno a capo}
	Nel linguaggio \SBF\ le istruzioni possono terminare con ';' oppure '\textbackslash n'. 
	\begin{equation*}
	\token{End} ::= \texttt{;}\ |\ \token{NewLine}
	\end{equation*}
	
	Il carattere di ritorno a capo '\textbackslash n' è sintatticamente significativo, e corrisponde a un token $\token{NewLine}$ quando le seguenti condizioni sono soddisfatte:
	\begin{itemize}
		\item il token immediatamente precedente può terminare una istruzione;
		\item il token immediatamente successivo può iniziare una istruzione;
		\item il carattere appare in una regione dove il new line è abilitato.
	\end{itemize}
	\hl{Complicate da spiegare, sono scritte qua. }
	\url{https://scala-lang.org/files/archive/spec/2.13/01-lexical-syntax.html#identifiers}
	
	\subsection*{Letterali}
	Ci sono letterali per numeri interi, numeri in virgola mobile, singoli caratteri, booleani, stringhe. Essi seguono le convenzioni della maggior parte dei linguaggi di programmazione.
	\begin{align*}
	\token{Literal} ::=
	& \token{Int}\\
	& \token{Float}\\
	& \token{Char}\\
	& \token{Boolean}\\
	& \token{String}
	\end{align*}
	
	\subsection*{Caratteri di spaziatura e commenti}
	I token possono essere separati da caratteri di spaziatura o commenti.
	
	I commenti sono di due tipi:
	\begin{itemize}
		\item i commenti di una riga sono sequenze di caratteri che iniziano con \verb$\\$ e finiscono al termine della riga;
		\item i commenti multi-riga sono sequenze di caratteri che iniziano con \verb$/*$ e terminano con \verb$*/$. Possono essere annidati, ma in modo bilanciato. Ad esempio \verb$/*commento*/*/$ non è un commento valido.
	\end{itemize}
	
	\hl{Permettiamo le trailing commas? In caso va detto qua in una sottosezione che non diventano token.}
	
	\section{Struttura sintattica di \SBF}
	\begin{itemize}
	\item Un \emph{programma} è una sequenza di dichiarazioni.
	
	\item Una \emph{dichiarazione} ha una delle seguenti forme:
	\begin{itemize}
		\item \emph{Dichiarazione di variabili.} \hl{Descrizione}
		\item \emph{Dichiarazione di funzioni/procedure.} \hl{Descrizione}
	\end{itemize}

	\item Un \emph{blocco} ha la forma
	\begin{align*}
	\token{Block} ::=\ &\texttt{\{} \token{Stmts}\ \token{Expr} \texttt{\}}\\
	|\ & \texttt{\{} \token{Stmts} \texttt{\}}
	\end{align*}
	Il primo elemento $\token{Stmts}$ è una sequenza, eventualmente vuota, di istruzioni.
	
	\item Una \emph{istruzione} ha la forma
	\begin{align*}
	\token{Stmt} ::= \ & \token{Block}\\
	| \ & \token{Ident}\ \texttt{(} \ \token{RExprList}\ \texttt{)}
	\end{align*}
	
	\end{itemize}
	
\end{document}
