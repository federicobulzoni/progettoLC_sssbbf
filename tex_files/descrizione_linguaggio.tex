\section{Descrizione del linguaggio \SBF}
Nella descrizione del linguaggio sono evidenziate solo le caratteristiche salienti, omesse le scelte standard e le richieste presenti nella consegna.

\section{Struttura lessicale di \SBF}

\subsection*{Parole riservate}
Le parole riservate in \SBF\ sono le seguenti:
\begin{gather*}
\If,\ \Else,\ \Do,\ \While,\ \Def,\ \Return,\ \Var,\ \texttt{Array},\ \False,\ \True,\\
\texttt{Null},\ \texttt{Char},\ \texttt{String},\ \texttt{Int},\ \texttt{Float},\ \texttt{Bool}.
\end{gather*}

\subsection*{Identificatori}

Un identificatore $\token{Ident}$ è una lettera seguita da una sequenza arbitraria di lettere, cifre e del carattere '\_'.

\subsection*{Letterali}
Vi sono letterali per numeri interi, numeri in virgola mobile, singoli caratteri, booleani, stringhe. Essi seguono le convenzioni della maggior parte dei linguaggi di programmazione.
\begin{align*}
\token{Literal} ::=\ 
& \token{Int}\\
|\ & \token{Float}\\
|\ & \token{Char}\\
|\ & \token{Bool}\\
|\ & \token{String}\\
|\ & \texttt{Null}
\end{align*}

\subsection*{Commenti}
I commenti in \SBF\ sono di due tipi:
\begin{itemize}
	\item i commenti di una riga sono sequenze di caratteri che iniziano con \verb$\\$ e finiscono al termine della riga;
	\item i commenti multi-riga sono sequenze di caratteri che iniziano con \verb$/*$ e terminano con \verb$*/$. Non possono essere annidati.
\end{itemize}

\subsection*{Caratteri di spaziatura}
I token possono essere separati dai caratteri di spaziatura standard o commenti.

\section{Struttura sintattica di \SBF}
\begin{itemize}
	\item Un \emph{programma} è una sequenza di dichiarazioni.
	
	\item Una \emph{dichiarazione} ha una delle seguenti forme:
	\begin{itemize}
		\item \emph{Dichiarazione di variabili}
		\begin{align*}
		\token{Decl} ::=\ &\texttt{var}\ \token{Ident} \texttt{:}\ \token{TypeSpec}\ \texttt{=}\ \token{Expr}\ \texttt{;}\\
		|\ &\texttt{var}\ \token{Ident} \texttt{:}\ \token{TypeSpec} \texttt{;}
		\end{align*}
		
		\item \emph{Dichiarazione di funzioni e procedure}
		\begin{align*}
		\token{Decl} ::=\ & \texttt{def}\ \token{Ident}\ \token{ParamClauses} \texttt{:}\ \token{TypeSpec}\ \texttt{=}\ \token{Expr}\ \texttt{;}\\
		& \texttt{def}\ \token{Ident}\ \token{ParamClauses} \texttt{:}\ \token{TypeSpec}\ \texttt{=}\ \token{Block}\\
		& \texttt{def}\ \token{Ident}\ \token{ParamClauses}\ \texttt{=}\ \token{Expr}\ \texttt{;}\\
		& \texttt{def}\ \token{Ident}\ \token{ParamClauses}\ \texttt{=}\ \token{Block}
		\end{align*}
		
	\end{itemize}
	dove $\token{TypeSpec}$ è una specifica di tipo, che ha la forma
	\begin{align*}
	\token{TypeSpec} ::=\ &\token{SimpleType}\\
	|\ &\texttt{*}\ \token{TypeSpec}\\
	|\ &\texttt{Array}\ \texttt{[}\ \token{TypeSpec}\ \texttt{]}\ \texttt{(}\ \token{Int}\ \texttt{)}\\
	\token{SimpleType} ::=\ &\texttt{Bool}\ |\ \texttt{Char}\ |\ \texttt{Int}\ |\ \texttt{Float}\ |\ \texttt{String}
	\end{align*}
	mentre l'elemento $\token{ParamClauses}$ è una sequenza, non vuota, di $\token{ParamClause}$, e ciascun $\token{ParamClause}$ ha la forma
	\begin{equation*}
	\token{ParamClause} ::=\ \texttt{(}\ \token{Args}\ \texttt{)}
	\end{equation*}
	$\token{Args}$ è una sequenza, che può essere vuota, di elementi separati da virgola della forma
	\begin{equation*}
	\token{Arg} ::=\ \token{Ident}\ \texttt{:}\ \token{TypeSpec}
	\end{equation*}
	
	Ad esempio, una specifica di tipo valida è $\texttt{Array[*Int](2)}$, che indica un array di puntatori ad interi di dimensione $2$. Ad esempio, una definizione di funzione valida è
	\begin{verbatim}
	def foo(a: Array[*Int](2),p: *Int)(x: Int): Int = 
		*a[1] + *a[2] + *p + x;
	\end{verbatim}
	Essa prende come parametri un array di puntatori ad interi, un puntatore ad un intero e un intero e restituisce un intero.
	
	\item Un \emph{blocco} è una sequenza di istruzioni racchiuse fra parentesi graffe.
	\begin{equation*}
	\token{Block} ::=\ \texttt{\{}\ \token{StmtList}\ \texttt{\}}
	\end{equation*}
	
	\item Una \emph{istruzione} ha la forma:
	\begin{align*}
	\token{Stmt} ::=\ &\token{Decl}\\
	|\ &\token{Block}\\
	|\ &\token{LExpr}\ \token{OpAssign}\ \token{Expr}\ \texttt{;}\\
	|\ &\texttt{if}\ \texttt{(}\ \token{Expr}\ \texttt{)}\ \token{Stmt}\\
	|\ &\texttt{if}\ \texttt{(}\ \token{Expr}\ \texttt{)}\ \token{Stmt}\ \texttt{else}\ \token{Stmt}\\
	|\ &\texttt{while}\ \texttt{(}\ \token{Expr}\ \texttt{)}\ \token{Stmt}\\
	|\ &\texttt{do}\ \token{Stmt}\ \texttt{while}\ \texttt{(}\ \token{Expr}\ \texttt{)}\ \texttt{;}\\
	|\ &\texttt{return}\ \texttt{;}\\
	|\ &\texttt{return}\ \token{Expr}\ \texttt{;}\\
	|\ &\token{Ident}\ \token{ParamsList}\ \texttt{;}
	\end{align*}
	dove, nell'ultima istruzione, che corrisponde alla chiamata di procedura o funzione, $\token{ParamsList}$ è una sequenza, non vuota, di $\token{Params}$ della forma
	\begin{equation*}
	\token{Params} ::=\ \texttt{(}\ \token{ExprList}\ \texttt{)}
	\end{equation*}
	e $\token{ExprList}$ è una sequenza, che può essere vuota, di $\token{Expr}$ separate da virgola.
	
	Invece $\token{OpAssign}$ è uno dei seguenti operatori di assegnamento:
	\begin{equation*}
	\token{OpAssign} ::=\ \texttt{=}\ |\ \texttt{+=}\ |\ \texttt{-=}\ |\ \texttt{*=}\ |\ \texttt{/=}\ |\ \texttt{\%=}
	\end{equation*}
	
	\item Le \emph{left expressions} del linguaggio hanno la seguente forma:
	\begin{align*}
	\token{LExpr} ::=\ &\token{Ident}\\
	|\ &\token{LExpr}\ \texttt{[}\ \token{Expr}\ \texttt{]}\\
	|\ &\texttt{*}\ \token{LExpr}\\
	|\ &\texttt{(}\ \token{LExpr}\ \texttt{)}
	\end{align*}
	L'operatore accesso ad array $\texttt{[]}$ ha la precedenza sull'operatore di dereference $\texttt{*}$. Quindi ad esempio $\texttt{*a[1]}$ è sintatticamente equivalente a $\texttt{*(a[1])}$.
	
	\item Le \emph{right expressions} del linguaggio hanno la seguente forma:
	\begin{align*}
	\token{Expr} ::=\ &\token{Literal}\\
	|\ &\token{LExpr}\\
	|\ &\texttt{\&}\ \token{LExpr}\\
	|\ &\texttt{Array}\ \texttt{(}\ \token{ExprList}\ \texttt{)}\\
	|\ &\token{Ident}\ \token{ParamsList}\\
	|\ &\token{Expr}\ \token{BinOp}\ \token{Expr}\\
	|\ &\token{UnOp}\ \token{Expr}\\
	|\ &\texttt{(}\ \token{Expr}\ \texttt{)}
	\end{align*}
	\begin{align*}
	\token{BinOp} ::=\ 
	&\texttt{||}\ |\ \texttt{\&\&}\ |\ \texttt{<}\ |\ \texttt{<=}\ |\ \texttt{>}\ |\ \texttt{>=}\ |\ \texttt{==}\ |\  \texttt{!=}\ 
	|\ \texttt{+}\ |\ \texttt{-}\ |\ \texttt{*}\ |\ \texttt{/}\ |\ \texttt{\%}\ |\ \texttt{\^{}} \\
	\token{UnOp} ::=\ &\texttt{!}\ |\ \texttt{-}
	\end{align*}
	Gli operatori hanno precedenze e associatività standard.
\end{itemize}

\section{Vincoli di semantica statica di \SBF}
\subsection*{Scoping}
Il linguaggio ha scoping statico con visibilità dal punto di dichiarazione in poi. Le regole per la visibilità di dichiarazioni locali sono le usuali.

Gli identificatori di variabili e funzioni/procedure devono essere unici. Lo spazio dei nomi è unico, ossia non è permesso dichiarare una variabile e una funzione/procedura con lo stesso nome. Ovviamente questo vale per identificatori con visibilità non disgiunta.

\subsection*{Vincoli riguardanti i tipi}
Il linguaggio ha 5 tipi base e 2 costruttori di tipo come indicato nella definizione di $\token{TypeSpec}$. Non sono previste compatibilità fra tipi diversi. Il valore $\texttt{Null}$ può essere assegnato solo a left-expression il cui tipo ha un puntatore al top-level.

I tipi degli operatori binari sono i seguenti.
\begin{table}[h]
	\centering
	\begin{tabular}{l|l}
		$\texttt{+},\ \texttt{-},\ \texttt{*},\ \texttt{/},\ \texttt{\^{}},\ \texttt{\%} $ & $\texttt{Int} \times \texttt{Int} \to \texttt{Int}$ o $\texttt{Float} \times \texttt{Float} \to \texttt{Float}$\\ \hline
		$\texttt{||},\ \texttt{\&\&} $ & $\texttt{Bool} \times \texttt{Bool} \to \texttt{Bool}$ \\ \hline
		$\texttt{<},\ \texttt{<=},\ \texttt{>},\ \texttt{>=} $ & $\texttt{Int} \times \texttt{Int} \to \texttt{Bool}$ o $\texttt{Float} \times \texttt{Float} \to \texttt{Bool}$\\ \hline
		$\texttt{==},\ \texttt{!=}$  & $\tau \times \tau \to \texttt{Bool}$ dove $\tau$ è un qualsiasi tipo
	\end{tabular}
\end{table}

I tipi degli operatori unari sono i seguenti.
\begin{table}[h]
	\centering
	\begin{tabular}{l|l}
		$\texttt{!}$ & $\texttt{Bool} \to \texttt{Bool}$\\ \hline
		$\texttt{-}$ & $\texttt{Int} \to \texttt{Int}$ o $\texttt{Float} \to \texttt{Float}$
	\end{tabular}
\end{table}

Le guardie di istruzioni $\texttt{if}$, $\texttt{do-while}$ e $\texttt{while}$ devono avere tipo $\texttt{Bool}$.

\subsection*{Altri vincoli}
\hl{TODO. Return.}