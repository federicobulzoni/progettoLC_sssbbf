\subsection{Gestione delle istruzioni TAC tramite la monade State}

Per la gestione della creazione delle istruzioni TAC durante la creazione di quest'ultimo, è stata utilizzata la monade {\tt State a}. Questa permette di manipolare l'inserimento di nuove istruzioni, la creazione di nuove etichette e di nuovi temporanei, in modo equiparabile ad un cambio di stato in una computazione. La creazione e l'inserimento di una nuova istruzione nel codice TAC corrispongono, appunto, ad una modifica dello stato corrente, La monade {\tt State a} utilizza due funzioni per operare su quest'ultimo: {\tt get}, per ottenere lo stato, e {\tt put} per aggiornarlo. 

La monade {\tt State a} utilizzata durante la creazione del Three Address Code opera su stati formati da tuple di quattro elementi:
\begin{itemize}
    \item un intero che indica l'indice dell'ultimo temporaneo creato;
    \item un intero che indica l'ultima etichetta creata;
    \item la lista di istruzioni TAC che compongono il TAC in output;
    \item una lista di liste di istruzioni TAC. Ogni lista contiene un insieme di istruzioni di una singola funzione. 
\end{itemize}

Lo stato viene modificato nel momento in cui: si crea una nuova etichetta e/o un nuovo temporaneo tramite le funzioni {\tt newLabel} e {\tt newTemp}, rispettivamente (per evitare duplicazioni tra le istruzioni) e quando una nuova istruzione TAC di una funzione viene inserita nella lista delle istruzioni di quella funzione (tramite la funzione {\tt out}). Inoltre, vi è una modifica dello stato nel momento in cui tutti gli statetement di una funzione sono stati tradotti in istruzione TAC: l'insieme delle istruzioni di quella funzione vengono aggiunte in testa all'insieme delle istruzioni TAC principale (quelle che compongono il TAC in output), tramite la funzione {\tt pushCurrentStream}.

Nel file {\tt ThreeAddressCode.hs} è stato utilizzato {\tt TacState a} come alias per {\tt State (Int, Int, [TAC], [[TAC]]) a}, dove {\tt TAC} è un'istruzione TAC.