\subsection{Descrizione soluzione - Three Address Code}

Il Three Address Code (TAC) viene costruito a partire dall'albero annotato risultante dall'analisi di semantica statica.Per la costruzione viene fatto largo uso della monade State, descritta nella sezione precedente.
Nel TAC gli indirizzi utilizzati nelle istruzioni sono:
\begin{itemize}
    \item Letterali: per valori costanti di tipo base.
    \item Indirizzi di variabile e di funzioni: hanno la seguente forma,
    \begin{center}
        {\tt ident@loc}
    \end{center}
    dove {\tt ident} è il loro identificatore e {\tt loc} la locazione di dichiarazione.
    \item Temporanei: utilizzati per identificare espressioni.
\end{itemize}
Inoltre, nel TAC sono presenti le etichette che vengono utilizzate per indicare una locazione univoca nel flusso di esecuzione. Nel caso delle funzioni, come etichetta, viene utilizzato il loro indirizzo.
%%%%%%%%%%%%%%%%%%%%%%%%%%%%%%%%%%%%%% genProg, genDecls, genDecl

La costruzione del Three Address Code inizia dalla funzione {\tt genProg} che ha i seguenti compiti:
\begin{itemize}
    \item Inserire un'etichetta in fondo a tutte le istruzioni (e il relativo {\tt Call} dopo le dichiarazioni globali) se durante l'analisi di semantica statica non è stato trovato una funzione main da cui far partire la computazione. In alternativa, viene inserita un'istruzione {\tt Goto} all'etichetta del main;
    \item iniziare la discesa nell'albero annotato per poter costruire le singole istruzioni, tramite la funzione {\tt genDecls}.
\end{itemize}

La funzione {\tt genDecls} prende in input una lista di dichiarazioni e le scorre producendo le relative istruzioni TAC. Questo avviene tramite la funzione {\tt genDecl}; questa divide le dichiarazioni nei quattro casi possibili:
\begin{itemize}
    \item Dichiarazione di variabile con assegnamento: viene creato un nuovo indirizzo a partire dall'identificatore della variabile e dalla sua locazione di dichiarazione. Gli viene assegnato un valore tramite la funzione {\tt genExpAddr} che permette di evitare la creazione di temporanei inutili quando ci troviamo nel caso di un'espressione semplice (es. $x = 3$, $*y= x$, $x = y + z$ ecc.) ed esegue i necessari cast.
    \item Dichiarazione di variabile: come nel caso precente costruiamo un indirizzo a partire da identificatore e locazione, e gli assegnamo un valore di default (\label{table:valori_default}) tramite la funzione {\tt buildDefaultValue}.
    \item Dichiarazione di funzione/procedura: in questo caso viene creata una nuova etichetta per la funzione. Viene chiamata la funzione {\tt genBlock} che (tramite {\tt genStms}) controlla la presenta dell'istruzione {\tt return} (nel caso delle procedure verrà aggiunto un'istruzione di return vuoto) e costruisce tutte le istruzioni TAC relative al corpo della funzione. Al termine di questo processo le istruzioni TAC che sono state create vengono estratte tramite operazione di pop dalla lista di liste di istruzioni dello stato e inserite in testa al codice globale del TAC. Questo viene fatto per far sì che una funzione dichiarata nel corpo di un'altra venga, inserita sopra quest'ultima a livello di Three Address Code, e non al suo interno. Inoltre, viene controllato che se la funzione che sta venendo dichiarata è la funzione {\tt main} nello scope globale, allora essa deve avere una firma specifica: {\tt def main()}; quindi deve essere una procedura che non richiede alcun parametro.
\end{itemize}

\begin{table}
\centering
    \begin{tabular}{|c c|} 
        \hline
        Tipo & Valore di default \\ [0.5ex] 
        \hline\hline
        Int & 0 \\ 
        Float & 0.0  \\
        Char & \textbackslash0 \\
        String &  "" \\
        Bool & False \\ 
        Pointer & Null\\
        Array & - \\[1ex] 
        \hline
    \end{tabular}
    \caption{Valori di default}
    \label{table:valori_default}
\end{table}


%%%%%%%%%%%%%%%%%%%%%%%%%%%%%%%%%%%%%% genBlock, genStms, genStm

Ogni funzione è costituita da un insieme di statement (contenuti in un blocco). La funzione che si occupa di percorrere tutti gli statement di un blocco è {\tt genStms}, che restituisce {\tt True} se l'ultimo statement è un {\tt return} (per evitare che nel TAC compaiano due istruzioni {\tt return} consecutive). Nel contempo vengono esaminati gli statement singolarmente tramite la funzione {\tt genStm}.
Quest'ultima prende in esame tutti i possibili statement:
\begin{itemize}
    \item Dichiarazione: viene riutilizzata {\tt genDecl} per la gestione delle dichiarazioni interne al blocco.
    \item Blocchi interni: viene riutilizzata la funzione {\tt genBlock}.
    \item Assegnamenti: un assegnamento ha la seguente forma:
    \begin{center}
        {\tt Lexpression = RExpression}
    \end{center}
    Le {\tt l-expression} comprendono: accessi ad array, variabili puntatori e identificatori di variabile. La loro gestione avviene tramite la funzione {\tt genLexp}. Le {\tt r-expression} oltre che le {\tt LExpression}, comprendono le espressioni.
    \item Chiamata di procedura: vengono create le istruzioni che indicano i parametri passati alla funzione (tramite la funzione {\tt genParams} che crea istruzioni {\tt Param addr}) e viene creata un'istruzione {\tt Call} del TAC. La funzione {\tt genParams} si occupa anche di deferenziare la variabile passata alla funzione se il metodo di passaggio richiesto è {\tt ref}, {\tt res}, {\tt valres}. Al momento della stampa, le funzione già definite nel linguaggio (es. {\tt writeInt}) avranno, al posto della locazione specificata nell'indirizzo, la parola {\tt default} (es. {\tt writeInt@default}).
    \item Return: se l'istruzione {\tt return } ritorna un'espressione essa viene gestita tramite {\tt genExpAddr}, che in caso di espressioni complesse (non binaria tra due identificatori o unaria) restituisce un temporaneo.
    \item Break e Continue: viene creata un'istruzione {\tt Goto} con associata un'etichetta che indica l'inizio del ciclo in caso di {\tt continue} e l'istuzione subito dopo il ciclo nel caso di {\tt break}.
    
    \item For: oltre che all'istruzione di controllo per le varie iterazioni, viene inserito in coda a tutte le istruzioni TAC relative al corpo del ciclo, l'istruzione di incremento dell'iteratore. 
    
    \item Do-While: viene modificato lo stato con le etichette relative ai {\tt break} e {\tt continue}. In questo caso le etichette create sono tre: quella relativa all'inizio del ciclo (utilizzata dalla condizione di iterazione), quella che indica la posizione del controllo dell'iterazione (utile per il {\tt Goto}  del {\tt continue}) e quella che indica l'istruzione successiva alla posizione di controllo (utile per il {\tt Goto} del {\tt break}). Al termine delle istruzioni relative al ciclo, il valore dello stato relativo alle etichette {\tt break} e {\tt continue} viene ristabilito al precedente valore;
    \item While: come nel caso precedente vengono create le etichette per il {\tt break} e il {\tt continue}, che, vengono gestite come nel caso precedente. In questo caso le etichette create sono solo due: quella che rimanda all'inzio del ciclo e quella che rimanda alla prima istruzione dopo il ciclo;
    \item If: banalmente vengono create le istruzioni relative agli statement relativi all'{\tt if} e all'{\tt else}. Se il corpo dell'{\tt else} è vuoto, allora l'etichetta {\tt labelElse} dopo le istruzioni del caso {\tt if} non viene generata.
\end{itemize}

Sia per quanto riguarda gli {\ff If} e i cicli ({\tt For, Do-While} e {\tt While}) la gestione delle etichette avviene tramite la funzione {\tt genCondition}, che, presa l'espressione della condizione e le due etichette che indicano dove spostarsi in caso essa sia vera o meno, gestitisce la creazione dei {\tt Goto} ed esegue il controllo del flusso. La funzione {\tt genCondition} viene utilizzata in particolar modo per evitare controlli superflui (es. nella condizione {\tt a \&\& b} se {\tt a} è falsa, non serve valutare anche {\tt b} e il controllo può passare all'etichetta relativa al caso {\tt False} della condizione).

%%%%%%%%%%%%%%%%%%%%%%%%%%%%%%%%%%%%%% genExp

La generazione delle istruzioni riguardanti le espressioni viene gestita tramite la funzione {\tt genExpAddr} che restituisce l'indirizzo TAC dell'espressione: un letterale nel caso di espressioni composte da un solo letterale o un nuovo temporaneo nel caso di dereferenziazioni, LExpression, accessi ad array o espressioni simili a quelle descritte nel caso dell'assegnamento. Se il tipo dell'espressione tipata restituita durante la fase di analisi statica è diverso (ma sicuramente compatibile) con quello della sottoespressione che si sta analizzando, allora viene eseguito un {\tt cast} implicito.

%%%%%%%%%%%%%%%%%%%%%%%%%%%%%%%%%%%%%% Conslusione

Terminate tutte le dichiarazioni presenti nello scope globale, vengono inserite in testa alla lista di istruzioni TAC tutte le dichiarazioni di variabili globali seguite dall'istruzione {\tt Goto} alla funzione {\tt main} se presente, oppure all'etichetta {\tt l0} in fondo a tutte le istruzioni (evitando così che tutte le altre funzioni vengano eseguite sequenzialmente).